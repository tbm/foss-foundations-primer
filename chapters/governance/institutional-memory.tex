
% SPDX-FileCopyrightText: © 2021 Martin Michlmayr <tbm@cyrius.com>

% SPDX-License-Identifier: CC-BY-4.0

\setchapterimage[9.5cm]{images/elephant}
\chapter{Stability and institutional memory}
\labch{institutional-memory}

Projects are in a constant flux.  It's not just the code that changes, but also the collaborators.  The project founder may leave after several years.  The activity level of collaborators can go up and down depending on the circumstances in their lives (such as becoming parents).

Organizations can contribute to the longevity of a project by providing stability and institutional memory.  While the people involved in an organization change over time, the organization itself can provide consistency through a stable approach supported by a common mission, guiding principles, and documented policies.

Best practices to support stability and institutional memory include:

\begin{itemize}

\item Clearly defined mission and vision

\item Written policies

\item Good record-keeping (such as keeping meeting minutes and archiving contracts)

\item Onboarding and orientation information for new board members

\end{itemize}

Revision-control systems make it possible to understand years later why a certain code change was made.  Mailing list discussions can similarly be used to track down the reasons why certain changes were made.

While the sensitive nature of some matters (such as legal issues) don't allow organizations to discuss and archive everything in public, similar principles can be applied to corporate records: maintain good internal documentation and records, and document why certain actions were taken.

\begin{kaobox}[frametitle=The role of KDE e.V. in maintaining institutional memory for the KDE community]

Adriaan de Groot, a member of the board of directors of KDE e.V., explains how the organization helps maintain an institutional memory:

\begin{quote}

Very few of the original developers are still involved day-to-day with the KDE community.  Having an institution that provides multi-generational stability is good.  As old developers fade from view, their knowledge isn't typically entirely lost.  We had a ``lost knowledge from long ago'' talk at Akademy 2020, the annual conference of the KDE community, specifically to carry forward some of the more obscure institutional bits.

\end{quote}

Additionally, he emphasizes the role of guiding principles:

\begin{quote}

We carry the \href{https://manifesto.kde.org/}{KDE manifesto} around; the manifesto is something the community agreed on, but it shapes the way KDE e.V. works by providing \textit{moral} guidance.  That means we have a `constitution' that anchors us to a particular
set of principles.

It's important to have stability in the organization and innovation in the community.

\end{quote}

\end{kaobox}

