
% SPDX-FileCopyrightText: © 2021 Martin Michlmayr <tbm@cyrius.com>

% SPDX-License-Identifier: CC-BY-4.0

\setchapterimage[9.5cm]{images/waterpass}
\chapter{Level playing field}
\labch{level-playing-field}

Open source projects are increasingly being formed and led by companies.  Some of these projects garner significant attention from other parties, such as individuals and other companies.  In addition to using the software, those parties want to contribute and shape the direction of the project.

Unfortunately, some projects that are originated by companies do not offer a level playing field: contributions from the company are treated more favorably than those from `outsiders'.  The company may also require a contributor license agreement that gives it broad rights to the contributions, further increasing its control over the project.

Good stewards recognize that you have to set a project free in order for it to flourish.  FOSS foundations constitute a vendor-neutral venue where open collaboration among participants can take place on an equal footing.  By moving a project to a neutral foundation, companies can signal that they are serious about creating a level playing field.

The project and foundation can put governance structures in place to ensure that no single corporation can exert undue influence over the technical direction of a project.

\begin{kaobox}[frametitle=Kubernetes and the Cloud Native Computing Foundation]

Kubernetes, the highly successful platform for managing containerized workloads and services, was started by Google.  When version 1.0 was released, Google moved the project to the Cloud Native Computing Foundation (CNCF), a foundation operating within the Linux Foundation.

While Google is still a major contributor, it's by no means the only driving force behind the project.  According to \href{https://k8s.devstats.cncf.io/d/9/companies-table}{developer data from CNCF}, major contributors include Red Hat, VMware, Microsoft, IBM, Huawei, and others.

\end{kaobox}

