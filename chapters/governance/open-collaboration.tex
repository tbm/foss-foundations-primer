
% SPDX-FileCopyrightText: © 2021 Martin Michlmayr <tbm@cyrius.com>

% SPDX-License-Identifier: CC-BY-4.0

\setchapterimage[9.5cm]{images/collaboration}
\chapter{Open collaboration}
\labch{open-collaboration}

Open source is associated with open collaboration.  Some projects rely on contributions from unpaid volunteers, other projects show a high level of corporate engagement, and many projects have a mix of volunteer and paid labor.  Contributors work together in an open, collaborative manner to advance the project.

The Linux Foundation, which hosts many projects and organizations, describes \href{https://www.linuxfoundation.org/blog/the-linux-foundation-its-not-just-the-linux-operating-system/}{five key principles} of open collaboration:

\begin{itemize}

\item Organizational neutrality: assets of a project, such as domain names and trademarks, are held by a neutral organization rather than a single individual or company.  While the organization owns the assets legally, the community decides how to use them through its governance model.

\item Clear separation of funding and participation: providing financial support to the organization does not give companies or other donors the right to influence the technical direction of the project.  The technical development is governed separately and all contributions are evaluated on their own merit.

\item Open governance: projects operate in a transparent manner and have neutral and clearly defined governance models which set out how participation works and how decisions are made.

\item Intellectual property clarity: the removal of uncertainty over licensing makes it easier to contribute and to use projects.

\item Commercial support ecosystem: projects can benefit when a commercial ecosystem forms around them, which creates employment opportunities and provides funding to support the project, such as through sponsorship of a developer conference.

\end{itemize}

While the exact nature and operations of organizations and projects differ, these principles apply quite widely to the type of open collaboration that is supported by FOSS foundations.

\begin{kaobox}[frametitle=FOSS foundations: protecting the definition of open collaboration]

Mike Milinkovich, the Executive Director of the Eclipse Foundation, \href{https://twitter.com/stephenrwalli/status/1358884485387808768}{explained} the role of FOSS foundations play in supporting open collaboration:

\begin{quote}

We foster collaboration in an openly governed way that protects the assets that are created for all of the downstream consumers. We are institutionally mandated to resist those who try to redefine what openness means in our communities.

I really believe that open source foundations are an absolutely integral part of protecting the definition of open collaboration, enabling open innovation, and making that sustainable for the long term.

\end{quote}

\end{kaobox}

