
% SPDX-FileCopyrightText: © 2021 Martin Michlmayr <tbm@cyrius.com>

% SPDX-License-Identifier: CC-BY-4.0

\setchapterimage[9.5cm]{images/birds}
\chapter{Conflict resolution}
\labch{conflict-resolution}

Open collaboration is a wonderful way to develop software, but conflicts are bound to happen at some point in any project of significant size.  There can be disagreements on technical decisions, the direction a project is heading, or other aspects of the project.

A solid governance structure helps to reduce and resolve conflicts by providing clear policies and decision-making processes.  The culture of a project also has a big impact, especially if it encourages friendly collaboration.

Organizations can also run into conflicts, including conflicts of interest, especially trade associations which have corporate members with different commercial interests.

Organizations employ a number of mechanisms to prevent conflicts in the first place.  \href{https://doi.org/10.1007/978-3-030-33742-1_11}{Weikert, Riehle and Barcomb} studied several FOSS foundations and identified five major areas of conflict prevention:

\begin{itemize}

\item Screening processes: new members and projects are only accepted when they pass a screening process.  Such a process can identify common interests and the motivation of potential members.  The technical, cultural and strategic fit of projects can be assessed.  Some organizations use an incubation process and new projects can only `graduate` once they meet certain criteria.

\item Governance structures and rules: foundations have formal governance structures and rules which are codified in their by-laws.  They can include:

\begin{itemize}

\item Transparent affiliations: contributors have to declare their corporate affiliations in public to highlight potential bias.

\item Decoupling funding from control: the funding of the organization is separate to the technical decision-making.

\item Representation limits: some organizations limit the number of people from the same company who can serve on the board of directors.

\item Independent entities: key staff of the foundation, such as the executive director, may not be employees or consultants of a member company.

\end{itemize}

\item Explicit strategies: organizations protect their culture and values through a number of explicit strategies, such as monitoring the behavior of participants.

\item Common interests: even though member companies often compete in the market place, common interests allow them to collaborate.  These common interests can be technical (e.g. a focus on technical merit over corporate agendas) or of a business nature (e.g. building a common platform against a dominant competitor).

\item Culture and values: shared values prevent bad behavior and contribute to the success.  Important values include openness, transparency, equality, and neutrality.

\end{itemize}

