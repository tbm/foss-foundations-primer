
% SPDX-FileCopyrightText: © 2021 Martin Michlmayr <tbm@cyrius.com>

% SPDX-License-Identifier: CC-BY-4.0

\setchapterimage[9.5cm]{images/stonehenge}
\chapter{Legal structure}
\labch{legal}

The majority of FOSS foundations operate as non-profit organizations.  While there are FOSS foundations in many countries, presently the majority are based in the US.  The most common legal structures used by US-based FOSS foundations are:

\begin{itemize}

\itemsep 0.50em

\item Charities: these organizations work toward a specific mission that has a benefit to the general public.

\item Trade associations: these organizations allow companies to advance common business interests.

\end {itemize}

These two legal structures are often referred to as 501(c)(3) (for charities) and 501(c)(6) (for trade associations), as a reference to the section of the Internal Revenue Code where they are defined.

Stormy Peters \href{http://stormyscorner.com/2008/08/501c-3-versus-6.html}{highlights} a number of differences between charities and trade associations.  For example, donations to charities are tax deductible (for certain donors in the US).  This is not the case for trade associations, although membership fees can be treated as business expenses.  Also, while charities are only permitted very limited lobbying, trade associations can engage in political lobbying that is related to member interests.

There are various legal structures in other countries that have been used to set up FOSS foundations.  The Document Foundation's use of a German \textit{Stiftung} is notable.  In part, it was used specifically to make certain elements of its rules unalienable, therefore reducing the risk that the Foundation's initial ideals would be changed over time.

\begin{kaobox}[frametitle=FOSS foundations: what's in a name?]

Organizations that support the activities of open-source projects are often called `FOSS foundations` (as in this primer).  However, the term `foundation` has a specific meaning in some jurisdictions and therefore some people prefer different terminology, like association or non-profit organization.

\end{kaobox}

