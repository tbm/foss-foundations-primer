
% SPDX-FileCopyrightText: © 2021 Martin Michlmayr <tbm@cyrius.com>

% SPDX-License-Identifier: CC-BY-4.0

\setchapterimage[9.5cm]{images/menu}
\chapter{Services}
\labch{services}

FOSS organizations aim to meet the needs of the projects they serve.  Umbrella organizations typically offer a range of services, broadly reflecting the project needs described in this primer (such as mentorship, asset stewardship, and event management).  Different projects use different services based on their needs.  Typically, projects can choose \textit{à la carte} from a menu of services.  Umbrella organizations usually take a percentage of project donations in order to provide their services.

Of course, not every organization offers everything.  For example, several organizations don't employ staff to write software.  This can be for a number of different reasons, including:

\begin{itemize}

\itemsep 0.50em

\item Philosophical reasons: an organization may wish to provide infrastructure for a project but stay out of development.
\item Lack of demand: the ecosystem has enough companies that pay for development, so there's simply no need to employ developers.
\item Organizational limitations: some organizations are not prepared for the paperwork involved with having employees.

\end{itemize}

\begin{kaobox}[frametitle=Example services and support programs]

The Linux Foundation offers a wide range of \href{https://www.linuxfoundation.org/en/projects/support-programs/}{support programs}.  Their core support programs include project operations, tooling, training, certification, and event management.  Additional support programs include legal and IP management, HR support, marketing operations, and community growth.

The Software Freedom Conservancy offers many \href{https://sfconservancy.org/projects/services/}{member project services}, including contract negotiation and execution, conference logistical support, community elections, basic legal advice and services, FLOSS copyright license enforcement, and fundraising assistance, along with leadership mentoring, advice, and guidance.

The Center for the Cultivation of Technology groups \href{https://techcultivation.org/#overview}{its services} into four areas: donations, budget management, asset stewardship, and expert network.

\end{kaobox}

