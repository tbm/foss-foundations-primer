
% SPDX-FileCopyrightText: © 2021 Martin Michlmayr <tbm@cyrius.com>

% SPDX-License-Identifier: CC-BY-4.0

\setchapterimage[9.5cm]{images/umbrella}
\chapter{Umbrella organizations}
\labch{umbrella}

An umbrella organization is an association that coordinates activities and pools resources.  In the context of open source, FOSS foundations that operate as umbrella organizations accept projects under their `umbrella`.

Projects that join an umbrella organization benefit from the organization's legal structure and can access services offered by the organization, such as operations (accounting, legal, etc.), conference management, mentorship, and more.  Umbrella organizations typically take a percentage from incoming donations to cover their operational expenses.

The term \textit{fiscal sponsorship} is often used in the context of umbrella organizations.  This term is confusing because it does \textit{not} mean that the organization provides financial sponsorship to a project (in fact, organizations take a percentage of project donations).

What fiscal sponsorship means is that the umbrella organization acts as the host of the project and performs administrative functions on behalf of the sponsored project.  Crucially, the organization takes on the responsibility of receiving and administering donations and of complying with all legal obligations.

Fiscal sponsorship reduces the need for creating a new organization because a project can use the existing infrastructure of the umbrella organization.  Essentially, projects with similar needs can join up via an umbrella organization instead of each creating their own entity.

There are \href{https://lwn.net/Articles/548542/}{two types} of fiscal sponsors:

\begin{itemize}

\itemsep 0.50em

\item Comprehensive fiscal sponsors: projects become fully integrated into the sponsor, similar to a merger and acquisition.

\item Grantor/grantee fiscal sponsors: projects don't join the fiscal sponsor but remain autonomous; fiscal sponsors accept funds on behalf of the project and provide these funds to the project in the form of grants.

\end{itemize}

\begin{kaobox}[frametitle=Center for the Cultivation of Technology: a grantor/grantee fiscal sponsor]

The \href{https://techcultivation.org/}{Center for the Cultivation of Technology} is an example of a grantor/grantee fiscal sponsor:

\begin{quote}

The Center for the Cultivation of Technology is a ``backend provider'' for the Free Software community. Affiliated projects use the legal entity to collect donations and to manage their budgets and assets, while we take care of organizational matters such as accounting, reimbursements and financial reporting.

The Center aims to become a lightweight scalable fiscal sponsorship ``host'', much like a ``payment processor for unincorporated projects''. But in most cases, we work closer with projects to identify funding opportunities, provide additional services (together with partner organizations), exchange knowledge about sustainable organizational development, and allow developers to focus on what they do best: write code.

\end{quote}

\end{kaobox}

Looking at FOSS foundations, there's a wide range of how close or loose the relationship between the project and the umbrella organization is.  Furthermore, some organizations expect projects to operate a certain way while others are more hands-off.  For example, the Apache Software Foundation (ASF) requires projects to use the Apache license and follow the \href{https://www.apache.org/theapacheway/}{Apache Way}, its governance process and philosophy.  Software in the Public Interest, Inc., on the other hand, puts few restrictions on how projects operate.

Each of these approaches have their unique pros and cons.  Different projects have different needs and some FOSS foundations will fit their needs better than others.

\begin{kaobox}[frametitle=Debian and its trusted organizations]

Instead of operating a Debian Foundation, the Debian project works with several organizations (known as \href{https://wiki.debian.org/Teams/Treasurer/Organizations}{trusted organizations}) that provide services and hold assets for the project.  The project also has an agreement with Software Freedom Conservancy that lets \href{https://sfconservancy.org/copyleft-compliance/#debian}{contributors assign copyrights} if they wish to do so.

While this may be an unusual example, it shows that a loose relationship with one (or more) organizations works for some projects; other projects may benefit from a closer, integrated relationship with one organization (possibly even their own organization).

\end{kaobox}

