
% SPDX-FileCopyrightText: © 2021 Martin Michlmayr <tbm@cyrius.com>

% SPDX-License-Identifier: CC-BY-4.0

\setchapterimage[9.5cm]{images/dance}
\chapter{Culture}
\labch{culture}

The culture of a project is an important factor that influences the health and growth of a community.  Increasingly, projects put a great emphasis on creating a culture in which participants are welcome and can thrive.  The culture of a project often determines whether contributors will stick around.

Many projects publish a formal code of conduct, so participants know what to expect.  Such documents can emphasize the shared values as well as expected norms of the community.

The \href{https://www.contributor-covenant.org/}{Contributor Covenant} is a code of conduct which has been adopted by many projects and communities, including Jekyll, Jenkins, and the Linux kernel.

\begin{kaobox}[frametitle=Contributor Covenant]

Some benefits of adopting a code of conduct like the Contributor Covenant:

\begin{quote}

Adopting Contributor Covenant helps makes your community's values explicit, and signals your commitment to creating a welcoming and safe environment for everyone.

\end{quote}

\end{kaobox}

A code of conduct is only effective when community leaders follow it by example.  There must also be an enforcement mechanism to make sure collaborates adhere to it.  Problems with participants are often addressed through a formal process led by an organization.

\begin{kaobox}[frametitle=Django: enforcement of code of conduct]

The Django project has an \href{https://www.djangoproject.com/conduct/enforcement-manual/}{enforcement manual} which clearly sets out how the project responds to an issue.  Conduct violations are managed by a code of conduct committee, which is established by the Django Software Foundation.

\begin{quote}

The committee will then review the incident and determine, to the best of their ability:

\begin{itemize}

\item what happened
\item whether this event constitutes a code of conduct violation
\item who, if anyone, was the bad actor
\item whether this is an ongoing situation, and there is a threat to anyone's physical safety

\end{itemize}

\end{quote}

\end{kaobox}

