
% SPDX-FileCopyrightText: © 2021 Martin Michlmayr <tbm@cyrius.com>

% SPDX-License-Identifier: CC-BY-4.0

\setchapterimage[9.5cm]{images/market}
\chapter{Marketing and promotion}
\labch{promotion}

Many open-source developers have a ``if you build it, they will come'' mentality, but active promotion can lead to more visibility.  The more visible a project is, the more developers and users it can reach.  Promotional activities include giving presentations at conferences, participating in podcasts, speaking to journalists, building an informative website, recording screencasts, and more.  Such activities can target developers, users, or both.

In addition to promotion, a large part of marketing is understanding the target market better.  Conferences are a good way for developers to interact with users to hear how they use the software and what problems they face.  Meeting users and listening to interesting use cases can be a great source of motivation for developers.

Some projects have outreach programs in order to increase the size of the community, including cooperation with universities and mentorship programs.

FOSS foundations can assist with promotional activities in a number of ways.  They can purchase merchandise for conferences, take care of the logistics of paid mentorship programs, negotiate contracts with graphical designers, and more.  Many FOSS foundations have a good network of connections and can open new doors.  Joining a prominent foundation can also give a project a boost in terms of visibility.

Finally, some FOSS foundations play an advocacy role, promoting ideas like software freedom and open standards.

\begin{kaobox}[frametitle=Marketing efforts of LibreOffice]

LibreOffice's \href{https://www.libreoffice.org/community/marketing/}{marketing team} leads a number of efforts, including:

\begin{itemize}

\item Communications, such as press releases

\item Informing the community about what's going on inside the project

\item Development of collateral

\item Events management and participation

\item Social media presence

\item Development of specific promotional activities

\end{itemize}

\end{kaobox}

