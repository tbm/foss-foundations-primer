
% SPDX-FileCopyrightText: © 2021 Martin Michlmayr <tbm@cyrius.com>

% SPDX-License-Identifier: CC-BY-4.0

\setchapterimage[9.5cm]{images/egg}
\chapter{Incubation}
\labch{incubation}

Several FOSS organizations offer an incubation process, which is a formalized intake process for new projects (in particular for young projects).  The aim of such an incubation process is to help projects attain more maturity in a number of ways.

Incubation can cover different aspects of a project, including the creation of well-defined governance structures with the help of experienced members from the organization.  It can also involve mentorship and guidance on the diverse set of issues that a project may face.

Some projects can benefit from joining an umbrella organization: not only can they tap into considerable experience, but becoming part of a larger organization may also allow them to attract more attention quickly thanks to the brand of the organization.

\begin{kaobox}[frametitle=Apache Incubator]

The \href{https://apache.org/}{Apache Software Foundation} (ASF) offers an \href{https://incubator.apache.org/}{incubation process} for projects that wish to join the organization.  It helps incoming projects adopt the \href{https://apache.org/theapacheway/}{Apache Way}, Apache's style of governance and operation, and guides new projects to services the organization provides.

Apache provides an \href{https://incubator.apache.org/cookbook/}{Incubator Cookbook} that helps potential projects decide whether the ASF is a good fit for them and guides them through the steps required to join the incubation process.

\end{kaobox}

\begin{kaobox}[frametitle=CNCF Sandbox]

The \href{https://www.cncf.io/}{Cloud Native Computing Foundation} (CNCF) groups projects into three stages according to their maturity: sandbox, incubating, and graduated projects.

The \href{https://www.cncf.io/sandbox-projects/}{CNCF Sandbox} has four goals:

\begin{itemize}

\item Encourage public visibility of experiments and other early work

\item Facilitate alignment with existing projects

\item Nurture projects

\item Remove possible legal and governance obstacles to adoption and contribution

\end{itemize}

Projects can move to the next maturity level after demonstrating their sustainability according to clearly defined graduation criteria.

\end{kaobox}

