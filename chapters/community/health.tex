
% SPDX-FileCopyrightText: © 2021 Martin Michlmayr <tbm@cyrius.com>

% SPDX-License-Identifier: CC-BY-4.0

\setchapterimage[9.5cm]{images/stethoscope}
\chapter{Community health}
\labch{community-health}

Community health and growth can be measured, for example through metrics, and dashboards can be used to display and monitor the data.

\href{https://chaoss.community/}{Community Health Analytics Open Source Software} (CHAOSS) is a Linux Foundation project focused on creating analytics and metrics to help define community health.  It has working groups that focus on different areas, including common metrics, evolution, and diversity and inclusion.

\begin{kaobox}[frametitle=CHAOSS: the importance of community health]

\href{https://chaoss.community/about/}{CHAOSS} gives several reasons for monitoring community health:

\begin{quote}

\begin{itemize}

\item Open source contributors want to know where they should place their efforts and know that they are making an impact.
\item Open source communities want to attract new members, ensure consistent quality, and reward valuable members.
\item Open source companies want to know which communities to engage with, communicate the impact the organization has on the community, and evaluate the work of their employees within open source.
\item Open source foundations want to identify and respond to community needs, evaluate the impact of their work, and promote communities.

\end{itemize}

\end{quote}

\end{kaobox}

There are various dashboards available to show metrics of open-source projects.  One is \href{https://cauldron.io/}{Cauldron}, which aggregates information from multiple collaboration platforms.  Another is \href{https://bitergia.com/bitergia-analytics/}{Bitergia Analytics}, which is a project health dashboard that has been \href{https://bitergia.com/oss-foundations/}{adopted by a number of foundations} for the benefit of their member projects.

