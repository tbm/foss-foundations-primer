
% SPDX-FileCopyrightText: © 2021 Martin Michlmayr <tbm@cyrius.com>

% SPDX-License-Identifier: CC-BY-4.0

\setchapterimage[9.5cm]{images/stop}
\chapter{Patents}
\labch{patents}

Patents are a way to register an invention and get exclusive rights to the invention for a period of time.  They can be problematic for open source projects.  Projects generally don't patent their inventions, but instead they implement their ideas in software that they share widely.  However, projects can get attacked by companies that hold patents and allege infringement.  A patent defense can be extremely expensive because it involves lawyers.

The \href{https://openinventionnetwork.com/}{Open Invention Network} (OIN) plays an important role in protecting some open source projects by building up a pool of patents as a defensive strategy.  Projects and FOSS foundations can join OIN as \href{https://openinventionnetwork.com/our-members/community-members/}{community members}.

The Software Freedom Law Center has published a legal primer, which describes several \href{https://www.softwarefreedom.org/resources/2008/foss-primer.html#x1-390004}{defense mechanisms}.

\begin{kaobox}[frametitle=Patent case against GNOME]

The GNOME Foundation \href{https://foundation.gnome.org/2020/05/20/patent-case-against-gnome-resolved/}{successfully defended} itself against a patent infringement claim.  The outcome was remarkable: in addition to GNOME receiving a covenant not to be sued for any patent held by Rothschild Patent Imaging, they negotiated the same covenant for any software that is released under an open source license.

GNOME's defense was highly successful, but a positive outcome is not guaranteed.  Even if the project is successful in its defense, it will lose a lot of time that could have been spent on other activities, such as improving the software.

\end{kaobox}

