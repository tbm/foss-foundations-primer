
% SPDX-FileCopyrightText: © 2021 Martin Michlmayr <tbm@cyrius.com>

% SPDX-License-Identifier: CC-BY-4.0

\setchapterimage[9.5cm]{images/coke}
\chapter{Trademarks}
\labch{trademarks}

Trademarks play an important role in open-source projects.  Anthonia Ghalamkarizadeh, a lawyer, \href{https://lwn.net/Articles/546678/}{describes} a trademark as an ``identifier of origin``, which gives brand protection in two directions:

\begin{itemize}

\itemsep 0.50em

\item Protecting the reputation and values of the trademark owner

\item Protecting users from confusion and fraud by conveying a clear message about a product's origins

\end{itemize}

Unfortunately, popular open-source brands are sometimes used for nefarious purposes.  For example, downloads of `Firefox` were \href{https://lwn.net/Articles/546678/}{bundled with a costly subscription}, which users did not expect.  The Firefox trademark allows the project to remove illegitimate downloads that abuse its brand.

Trademarks are usually held by a FOSS foundation, which can take legal action if necessary.  Trademarks can be registered, and the \href{https://en.wikipedia.org/wiki/Madrid_system}{Madrid system} allows registration in multiple jurisdictions.  Registered trademarks need to be maintained and enforced.  The GNOME project, for example, had to \href{https://lwn.net/Articles/654124/}{defend its trademark} when Groupon wanted to use the name for an unrelated product.

It's important for projects to have trademark usage policies that clearly spell out acceptable and unacceptable uses of the mark.  The \href{http://modeltrademarkguidelines.org/}{Model Trademark Guidelines} are a good starting point for this.

\begin{kaobox}[frametitle=Trademark resources]

\href{https://fossmarks.org/}{FOSSmarks} offers a practical guide to understanding trademarks in the context of open-source projects, including basics on trademarks, choosing a name for a project, registering a name, and more.

The Apache Software Foundation maintains a \href{https://www.apache.org/foundation/marks/resources}{trademark page}, which lists not only its own trademark policy, but also contains a list of useful trademark information.

Finally, the Legal Issues Primer published by the Software Freedom Law Center contains a chapter on \href{https://www.softwarefreedom.org/resources/2008/foss-primer.html#x1-600005}{common trademark issues}.

\end{kaobox}

