
% SPDX-FileCopyrightText: © 2021 Martin Michlmayr <tbm@cyrius.com>

% SPDX-License-Identifier: CC-BY-4.0

\setchapterimage[9.5cm]{images/code}
\chapter{Licensing and copyright}
\labch{copyright}

The licensing of open-source projects is a widely discussed topic.  The choice of a license can greatly influence the impact and growth of a project.  Many resources exist that help projects make the right decision, such as the \href{https://choosealicense.com/}{choose an open source license} site.

In addition to the license itself, there are other aspects related to licensing and copyright that projects have to consider, and where FOSS foundations can play an important role.

Some projects require contributors to sign a Contributor License Agreement (CLA), which gives certain rights to the project.  If a CLA is required, a FOSS foundation would be a good home for the agreements.  However, CLAs are often seen as \href{https://opensource.com/article/19/2/cla-problems}{harmful} and the majority of projects rely on a system commonly referred to as \href{https://opensource.com/law/11/7/trouble-harmony-part-1}{inbound=outbound} whereby contributions to the project are provided under the project's license without the need for another agreement.  Some projects additionally use a simple self-attestation mechanism to confirm the origin of the contribution, such as the \href{https://developercertificate.org/}{Developer Certificate of Origin} (DCO) used by the Linux kernel and other projects.

Another area where FOSS foundations can help is the enforcement of open-source licenses.  While the idea of open source is to share the code widely, there are certain conditions attached to the distribution of the software.  Unfortunately, these are sometimes not followed (many times inadvertently, although sometimes intentionally).  FOSS organizations can work with those organizations to resolve any issues, or, if necessary, to initiate legal action.

Finally, while we don't like to think about death, it's important to plan ahead.  A project may want to change license or move to a new version of a license in the future, but this can only be done with the explicit permission of all copyright holders.  A FOSS organization could act as a custodian and make decisions for deceased contributors who granted them such permission.

