
% SPDX-FileCopyrightText: © 2021 Martin Michlmayr <tbm@cyrius.com>

% SPDX-License-Identifier: CC-BY-4.0

\setchapterimage[9.5cm]{images/woman-girl}
\chapter{Life cycle}
\labch{lifecycle}

The needs of projects change throughout their life cycles.  A young project typically has few needs (such as raising donations and holding a domain name).  As projects grow, so do their needs: they may want to apply for trademarks, organize conferences and developer meetings, hire contractors, and more.  Some projects that are part of an umbrella organization might also outgrow their host at some point and find enough reasons to justify setting up their own organizations.

Tony Sebro \href{https://lwn.net/Articles/548542/}{identified} three different roles that umbrella organizations sometimes play:

\begin{itemize}

\item Playpen: incubation for young projects.

\item Apartment: a long-term living arrangement for growing projects.

\item Nursing home: a place to put assets for inactive projects.

\end{itemize}

It's important for projects to consider what needs they currently have and how those needs might change in the future.

\begin{kaobox}[frametitle=jQuery and the jQuery Foundation]

The jQuery project \href{https://sfconservancy.org/news/2009/nov/30/jQuery-joins/}{joined} Software Freedom Conservancy, an umbrella organization, in 2009.  In 2012, Conservancy and the jQuery board \href{https://sfconservancy.org/news/2012/mar/06/jQuery-Foundation/}{announced} the creation of the jQuery Foundation.  The jQuery Foundation was created as a trade association with the aim to supporting the development of jQuery, provide documentation and support, and and foster the jQuery community:

\begin{quote}

``We are proud that the jQuery Board has built jQuery into a vibrant and successful Open Source community under Conservancy's mentorship,'' said Bradley M. Kuhn, Executive Director of the Software Freedom Conservancy. ``Our mission includes helping member projects determine whether to form their own organization, and we're pleased jQuery is the first Conservancy project to take that step.''

\end{quote}

\end{kaobox}

