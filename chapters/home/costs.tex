
% SPDX-FileCopyrightText: © 2021 Martin Michlmayr <tbm@cyrius.com>

% SPDX-License-Identifier: CC-BY-4.0

\setchapterimage[9.5cm]{images/heavy-load}
\chapter{Costs and administrative burden}
\labch{costs}

Developers greatly underestimate the administrative burden of running an organization.  There have been several examples over the years of FOSS foundations running into various operational issues, such as problems due to unfiled paperwork.

The Linux Foundation \href{https://www.linuxfoundation.org/blog/the-linux-foundation-its-not-just-the-linux-operating-system/}{observes} that ``overhead goes up exponentially`` and that ``maintainers need to deal with issues they never anticipated'':

\begin{itemize}

\itemsep 0.50em

\item Setting up a legal entity, which can take months of work and legal costs, then the ongoing maintenance of the entity and its associated state or federal filings

\item Setting up a bank account, obtaining checks, handling accounting, and establishing payroll, as well as administering benefits and hiring staff

\item Getting legal agreements signed by the sponsoring companies

\item Filling out paperwork and forms to be set up as a supplier with a sponsoring company

\item Dealing with invoices, purchase orders, and procurement departments

\item Setting up a financial reporting process, so stakeholders have visibility into where the funds went

\end{itemize}

It's important to take the costs and burden of running an organization seriously, to talk to lawyers and accountants to fully understand the responsibilities, and to think about the long-term implications and obligations.  For the majority of projects, the costs of creating a new organization outweigh the benefits.

It's therefore important to evaluate existing organizations first to see whether one of them would be a suitable home.

% Layout
\newpage

\begin{kaobox}[frametitle=Foundation as a service]

Umbrella organizations serve many individual projects, but what if several projects within a community want to join together as a foundation?

Several organizations follow a `foundation as a service' model whereby they provide operational support for other foundations.

The Linux Foundation is one example of such an organization.  It hosts initiatives and organizations, such as Let's Encrypt, the Cloud Native Computing Foundation (CNCF), and the OpenJS Foundation.  The Linux Foundation \href{https://www.linuxfoundation.org/join}{describes itself} as ``a federation of foundations'' and it ``provides a back-office infrastructure for projects and initiatives so they can focus on delivering value for their community``.

Software in the Public Interest, Inc.\ (SPI) is another organization that hosts (virtual) foundations.  The X.Org Foundation gave up its legal entity and became a virtual organization under SPI.  It kept its own governance structure (including its board) intact while using SPI as the legal entity.  The Open Bioinformatics Foundation also operates as a virtual organization with the support of SPI.

Finally, several FOSS foundations use the \href{https://opencollective.com/}{Open Collective} platform for crowdfunding for their projects.

\end{kaobox}

