
% SPDX-FileCopyrightText: © 2021 Martin Michlmayr <tbm@cyrius.com>

% SPDX-License-Identifier: CC-BY-4.0

\setchapterimage[9.5cm]{images/bridge}
\chapter{Open source as digital infrastructure}
\labch{digital-infrastructure}

Nadia Eghbal describes open source as digital infrastructure in her report \href{https://www.fordfoundation.org/work/learning/research-reports/roads-and-bridges-the-unseen-labor-behind-our-digital-infrastructure/}{Roads and Bridges}:

\begin{quote}

Much like roads or bridges, which anyone can walk or drive on, open source code can be used by anyone—from companies to individuals—to build software. This type of code makes up the digital infrastructure of our society today.

\end{quote}

Open source has fundamentally changed the way software is written.  Instead of starting a new software project entirely from scratch, programmers look for open source libraries that can be integrated to solve the problem at hand.  Furthermore, the advantages of open collaboration through open source is increasingly being recognized by companies, and they team up with other companies in order to share development costs and accelerate the pace of innovation.

Open source hasn't just had a huge impact on the development community.  Open source has become pervasive in our society, and we rely on it every day, even if we are not aware of it.  The Linux kernel forms the core of all Android phones and Kindle e-book readers, numerous WiFi routers, and many other products we use on a daily basis.  Other open source components are similarly widespread.

Modern life simply wouldn't be the same without thousands of open source libraries and programs written and maintained by a community of developers.  Of course, this tremendous success of open source also raises important questions --- since we rely on open source, how can we make its development more sustainable?

FOSS foundations, non-profit organizations that support open source projects, play an important role in contributing to the sustainability of open source.

