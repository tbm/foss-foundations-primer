
% SPDX-FileCopyrightText: © 2021 Martin Michlmayr <tbm@cyrius.com>

% SPDX-License-Identifier: CC-BY-4.0

\setchapterimage[9.5cm]{images/matryoshka}
\chapter{Growth issues}
\labch{growth}

It's easy to start an open-source project.  Popular platforms like GitHub and GitLab have reduced the barriers to entry even more.

The majority of projects stay limited in scope and reach.  However, many projects attract the attention of other developers and users, and they quickly grow beyond the original scope envisioned by the project originator.

Such success can be rewarding on many levels: helping others solve important problems, receiving feedback and code from others, interacting with a wide range of people, and more.

But success and growth can be a two-edged sword as they are also associated with additional work, increased responsibility, and tasks that one may not be interested in.

As a project grows, the structures supporting the project must adapt.  Decision-making is easy when you're the only one making decisions.  The larger the projects becomes, the more clearly defined the governance structures have to become.  How can others contribute to the project?  How are decisions made?  These are some of the questions that need to be answered and codified in a governance structure.

Successful projects also need to deal with issues they never expected, such as non-technical decisions and administrative work.  How can we accept donations?  What can we do with the money?  Do we need to register a trademark for our project name?

For the long-term health of the project, it's beneficial when the project's support structures don't rely on the project founder.  An organization, like a FOSS foundation, can provide an independent home for the project.

