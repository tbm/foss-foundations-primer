
% SPDX-FileCopyrightText: © 2021 Martin Michlmayr <tbm@cyrius.com>

% SPDX-License-Identifier: CC-BY-4.0

\setchapterimage[9.5cm]{images/tree}
\chapter{The role of foundations}
\labch{role}

While open-source projects come in all shapes and forms, most projects encounter a similar set of growth issues throughout their life cycles.

This primer covers non-technical aspects that the majority of projects will have to consider at some point.  It also explains how FOSS foundations can help projects.

While there are a number of different types of organizations, the emphasis in this primer is on umbrella organizations.  These are organizations which provide a nourishing environment and offer a set of services to projects that operate under their umbrella.

FOSS foundations play an important role in supporting open-source projects and contributing to their long-term sustainability.  The people involved in FOSS foundations have considerable experience and often have encountered similar problems to those a project faces.  They know from experience what works and what doesn't.  Some organizations also provide specific mechanisms to help young projects, for example through mentorship or a formal incubation process for new projects.

A foundation can also serve as an independent entity that represents the consensus of a project.  The process of creating a new organization or selecting an existing organization to join is a useful exercise because it forces a project to think about a number of issues, such as governance, asset management, culture, vision, and more.

There's a wide range of different FOSS foundations (as will be covered in more detail later).  They support open-source projects in number of key areas, including governance, community, financial, and legal.

