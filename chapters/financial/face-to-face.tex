
% SPDX-FileCopyrightText: © 2021 Martin Michlmayr <tbm@cyrius.com>

% SPDX-License-Identifier: CC-BY-4.0

\setchapterimage[9.5cm]{images/group_photo}
\chapter{Face-to-face meetings}
\labch{f2f}

There are many good reasons for projects to meet-to-face, such as hashing out technical solutions in an effective way, building social bonds, and increasing motivation for participants.

While it's fairly easy to meet at existing conferences, larger projects may wish to organize their own hackathons, developer meetings, or conferences.  This comes with organizational issues that are often not fully appreciated.  A venue needs to be selected and a contract signed, catering needs to be arranged, and an insurance policy should be taken out.  The project may wish to reimburse attendees for their travel expenses, which typically involves payments to many different countries.

Hopefully the event is a success, but it's important to plan for all eventualities:

\begin{itemize}

\item What if the conference is canceled due to a pandemic? Can the venue be canceled for free or will the project incur a huge venue expense despite the cancellation?
\item What if harassment happens at the event?  Are policies in place for that?

\end{itemize}

Some organizations hold annual events and are well versed with all organizational aspects of event management.  In fact, conferences are a substantial source of income for several organizations.

