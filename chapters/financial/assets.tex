
% SPDX-FileCopyrightText: © 2021 Martin Michlmayr <tbm@cyrius.com>

% SPDX-License-Identifier: CC-BY-4.0

\setchapterimage[9.5cm]{images/wheel}
\chapter{Asset stewardship}
\labch{assets}

Successful communities usually obtain a number of assets to support their projects, such as:

\begin{itemize}

\item Money

\item Servers and other technical equipment

\item Intangible assets, such as trademarks and domain names

\end{itemize}

Over the years, there have been several reports about projects running into problems because key assets were held by individuals who no longer represented the project's best interests.

Neutral asset stewardship is one of the great benefits an organization can offer to a project.  Instead of assets being owned by individuals involved in the project, all assets are owned by the organization.  The project, through its governance process, decides what to do with the resources; this happens in cooperation with the organization, which ensures that all legal obligations are met.

Asset stewardship involves many responsibilities: trademarks and domain names must be renewed periodically; projects expect financial reports about project funds; and an investment policy may be needed if an organization holds substantial funds.

\begin{kaobox}[frametitle=Projects with big balances]

Some prominent open source projects receive substantial donations and sponsorship, but often have few opportunities to actually spend the funds to improve the project.  Many collaborators are already paid by companies to work on the project, and only a small budget is needed to fund developer meetings and infrastructure.  Since expenses are lower than donations, a significant fund balance is built up over the years.

Donations are made to support the project.  Therefore, projects that accept donations should think about the best ways to use the funds to improve the project and community.

\end{kaobox}

