
% SPDX-FileCopyrightText: © 2021 Martin Michlmayr <tbm@cyrius.com>

% SPDX-License-Identifier: CC-BY-4.0

\setchapterimage[9.5cm]{images/keyboard}
\chapter{Development}
\labch{development}

Some projects pay directly for development work.  This can greatly increase the amount of work that can be done in some cases, but it can also introduce challenges and conflicts.

One potential issue relates to motivation.  Various studies have shown that a hobby that is turned into a job sometimes becomes a chore.  Many open-source contributors are driven by intrinsic motivation, such as a drive to learn and explore their potential.  External factors, such as money, are not as motivating as those that come from within.  Unfortunately, if external factors are introduced and later removed (e.g. funding stops), internal motivation may not fully come back, resulting in a loss of interest in contributing to the project.

There are also questions related to fairness.  If there's not enough money to pay all developers, a decision needs to be made as to whose work will be funded.  The project also needs to agree how much to pay, which can be difficult since salary expectations vary widely across the world.

Fortunately, many projects don't struggle with all of these problems.  It's quite common that developers don't seek funding for their work, but are happy to see other people's work being supported financially.  This can be because they already have a corporate sponsor (e.g. an employer who lets them work on the project) or because they prefer to stay in a volunteer capacity and work on the project as a hobby rather than a job.  Some projects also successfully fund tasks that nobody else is willing to take on.

If a project decides to fund development, some paperwork is necessary, such as a contract setting out expectations and responsibilities for everyone involved.  It's important to distinguish contracting work and employment, as the latter can lead to a lot of obligations on behalf of the project.  Many foundations have experience with getting contractors on-board for projects and doing all the necessary support work.

\begin{kaobox}[frametitle=Django Fellowship Program]

The Django Software Foundation runs the \href{https://www.djangoproject.com/fundraising/#fellowship-program}{Django Fellowship program}, which pays contractors to manage some of the administrative and community management tasks of the Django project:

\begin{quote}

The Django Fellowship program has a major positive impact on how Django is developed and maintained. The Django Fellows triage 10-15 new tickets each week and review and merge around fifteen non-trivial patches a week from the community. Release blocking and severe bugs aren't postponed indefinitely. Major releases happen according to an 8 month schedule and bug fix releases occur monthly.

\end{quote}

\end{kaobox}

