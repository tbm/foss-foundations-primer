
% SPDX-FileCopyrightText: © 2021 Martin Michlmayr <tbm@cyrius.com>

% SPDX-License-Identifier: CC-BY-4.0

\setchapterimage[9.5cm]{images/window}
\chapter{Transparency}
\labch{transparency}

Projects that accept donations have an obligation to be transparent about the ways the funds are being used.

Several projects and foundations publish annual reports that cover their activities for the year.  Contractors funded to work on open-source projects often publish monthly reports in which they document their accomplishments.

Many organizations also publish detailed financial reports about income and expenses.  Charities that are based in the US have to publish an annual tax filing (form 990) if they receive a certain amount of donations that year.  The tax filing contains information about the income and expenses, and it often gives a good overview of the activities of an organization.

Transparency also applies to other areas of an organization, such as the publication of by-laws and other governance documents.

\begin{kaobox}[frametitle=Transparency of FOSS foundations]

Many FOSS foundations publish annual reports, public filings, audited financial statements, and other materials.

Some examples include:

\begin{itemize}

\item Mozilla Foundation: \href{https://www.mozilla.org/en-US/foundation/annualreport}{annual reports, public filings, and audited financial statements}
\item NumFOCUS: \href{https://numfocus.org/community/mission/annual-reports}{annual reports} and \href{https://numfocus.org/legal}{public filings}
\item Open Infrastructure Foundation: \href{https://openinfra.dev/about/}{annual reports}
\item Software Freedom Conservancy: \href{https://sfconservancy.org/about/filings/}{public filings and audited financial statements}
\item The Documentation Foundation: \href{https://www.documentfoundation.org/foundation/financials/}{annual reports and accounting ledgers}

\end{itemize}

There's also a \href{https://gitlab.com/floss-foundations/npo-public-filings}{repository} where public filings from FOSS foundations are archived.

\end{kaobox}

