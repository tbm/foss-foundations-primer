
% SPDX-FileCopyrightText: © 2021 Martin Michlmayr <tbm@cyrius.com>

% SPDX-License-Identifier: CC-BY-4.0

\setchapterimage[9.5cm]{images/coins}
\chapter{Fundraising}
\labch{fundraising}

Significant projects typically need some funds to pay for expenses.  Even if the developers are not paid by the project, there are other expenses, such as domain names.

Projects can have a number of sources of funding, including:

\begin{itemize}

\item Donations (from individuals and corporations)
\item Merchandise (t-shirts and other items)
\item Sponsorship (for conferences or other activities)
\item Conference revenue (registration fees and sponsorship)
\item Royalties (for books or trademark usage)
\item Mentorship stipends (Google Summer of Code and Season of Docs)
\item Grants (from non-profit organizations or corporations)

\end{itemize}

While it's easy to put a PayPal button on a web site, there is considerable paperwork and administrative work involved with the majority of these funding sources.  Sponsorship and grants often involve legal agreements; many companies require a formal invoice and there are typically several steps associated with submitting invoices (registration as a vendor, creation of a purchase order, etc.); some donors wish to receive a written acknowledgment for their donation; and more.

Putting up a PayPal button also raises an important question: who does the money go to?  If a project has a single developer, putting up a personal PayPal link might be acceptable.  For larger projects, handling funds through a single developer is not an option.  The practice is also problematic from a tax perspective as accepting donations may create tax liabilities for the individual holding the funds.  This is where an independent organization steps in by keeping funds on behalf of the project and complying with all rules and regulations.  Finally, some donors may also prefer giving to an organization rather than to a specific developer.

Some organizations also have fundraising processes from which several projects benefit.  The brand and reputation of an established organization often makes it easier to attract funding.

